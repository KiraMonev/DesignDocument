\documentclass{article}
\usepackage{graphicx} % Required for inserting images
\usepackage[T2A]{fontenc}
\usepackage{geometry} % Managing page fields
\geometry{top=2cm, bottom=2cm, left=3cm, right=2cm}

\title{Дизайн документ}
\author{Авербух Семён, Бычкова Антонина, \\ Карташов Никита, Монёв Кирилл}


\date{Ноябрь 2024}

\begin{document}

\maketitle

\tableofcontents

\section{Введение}
Данный дизайн-документ знакомит с концепцией и спецификацией разрабатывающейся игры. Целью документа является создание структурированного и подробного описания множества аспектов игры, начиная от самой концепции и заканчивая техническими деталями реализации.

Игра разрабатывается как 2D-проект с видом сверху, где сюжет и сражения являются основным акцентом. Вдохновением для игры послужили такие проекты, как \textit{Hotline Miami} и \textit{Dead Cells}.

Цель игры — познакомить игрока с историей о борьбе и внутреннем росте главного героя, дать возможность поучаствовать в увлекательных сражениях и самому выбирать, как будут развиваться события. Игрок получит уникальный опыт взаимодействия с динамичным игровым миром, а также сможет проверить свои интеллектуальные способности.

\section{Концепция}
    \subsection{Введение}
    \subsection{Жанр и аудитория}
    \subsection{Основные особенности игры}
    \subsection{Описание игры}
    Главный герой — обычный парень, вынужденный противостоять алкашам, гопникам, тараканам и прочим персонажам, встречающимся на пути. Игрок перемещается по комнатам, участвуя в сражениях и исследуя окружающий мир. Игра представляет собой череду комнат и локаций, каждая из которых предлагает новые испытания.

    В начале игрок сталкивается с унылой обстановкой, но по мере прохождения и развития сюжета мир вокруг него меняется. Герой становится сильнее, открывает новые возможности и навыки. Атмосфера игры постепенно трансформируется, отражая внутреннюю эволюцию героя: от мрачных и приглушённых тонов — к насыщенным и жизнеутверждающим, что символизирует его рост и способность справляться с трудностями.
    \subsection{Предпосылки создания}
    \subsection{Платформа}

\section{Функциональная спецификация}
    \subsection{Принципы игры}
        \subsubsection{Суть игрового процесса}
        Игровой процесс в игре строится вокруг динамичных сражений и взаимодействия с окружающим миром. Игрок управляет главным героем, который постепенно становится сильнее, открывает новые возможности и проходит через испытания, преодолевая врагов и раскрывая сюжет.  
    
        Основное развлечение для игрока заключается в сочетании тактического выбора и быстрых, захватывающих схваток. Благодаря разнообразию врагов и уникальному подходу к сражениям, игроку придется адаптироваться к разным ситуациям, использовать холодное и огнестрельное оружие, а также наиболее эффективно взаимодействовать окружением.  
    
        В основе игрового процесса лежит идея, что каждый бой – это серьезное испытание. Игроку нужно учитывать слабости врагов, особенности своего оружия, а также расположение объектов на карте. Например, взрывоопасные бочки, укрытия или тесные проходы могут играть ключевую роль в сражении. Это добавляет глубину и заставляет игрока экспериментировать, подстраивая свою стратегию под новые условия.  
    
        Игра стремится создать опыт, который объединяет напряженность и удовольствие от победы. Каждый бой – это маленький вызов, который будет держать игрока в напряжении и позволять ему почувствовать себя настоящим героем, преодолевающим трудности.  
    
        Интерфейс будет интуитивно понятным, а управление – отзывчивым, чтобы максимально облегчить процесс вовлечения. Цель игры – подарить игроку эмоции: от чувства упадка в начале, до эйфории побед и прогресса в конце.  
    
        \subsubsection{Ход игры и сюжет}
        Игровой процесс делится на эпизоды, каждый из которых представляет собой набор комнат, связанных между собой. Каждая комната – это либо бой, либо интерактивная сцена, в которой можно собрать предметы, поговорить с NPC, или выполнить небольшой квест.  
    
        Типичный игровой сеанс начинается с того, что игрок оказывается в комнате, где его ждет какое-либо задание. Враги, такие как гопники, алкаши и даже тараканы, будут расставлены так, чтобы создавать вызовы. Игрок должен решить, каким оружием воспользоваться и как лучше всего подойти к ситуации.  
    
        На протяжении игры главный герой будет сталкиваться с новыми типами врагов, которые становятся умнее и опаснее. Каждый новый этап игры представляет усложнение: меняются как механики сражений, так и дизайн уровней. Например, узкие коридоры сменяются просторными помещениями, добавляются ловушки, а иногда герою придется использовать не только силу, но и смекалку.  
    
        Сюжет начинается с мрачной квартиры героя, где все буквально пропитано тоской и безысходностью. Первые враги – это тараканы и типичные бытовые раздражители, символизирующие внутреннюю апатию героя. 
    
        В процессе игрок узнает больше о прошлом героя, его мотивах и внутреннем состоянии. По мере продвижения обстановка меняется: унылые серые тона постепенно уступают место ярким краскам, окружающая среда становится все более красочной. Финальные эпизоды игры должны создать у игрока ощущение завершения пути: как будто он помог герою найти себя и изменить мир вокруг.  
    
        Каждая игровая сессия приносит не только прогресс, но и чувство достижения. Ключевая идея – показать, что путь преодоления трудностей, хотя и сложен, приносит внутреннюю награду. Это не только экшен, но и эмоциональная история, в которой игрок почувствует, что его действия имеют значение.  
    
        В игре будет много мелких деталей, которые оживляют мир: например, записки, оставленные на стенах, дневники NPC или небольшие шутки, спрятанные в диалогах. Все это должно создать ощущение, что мир игры живой, а история глубока и многослойна.


    \subsection{Физическая модель}
    \subsection{Персонаж игрока}
    \subsection{Элементы игры}
    \subsection{«Искусственный интеллект»}
    \subsection{Многопользовательский режим}
    \subsection{Интерфейс пользователя}
        \subsubsection{Блок-схема}
        \subsubsection{Функциональное описание и управление}
        \subsubsection{Объекты интерфейса пользователя}
    \subsection{Графика и видео}
        \subsubsection{Общее описание}
        \subsubsection{Двумерная графика и анимация}
        \subsubsection{Трехмерная графика и анимация}
        \subsubsection{Анимационные вставки}
    \subsection{Звуки и музыка}
        \subsubsection{Общее описание}
        \subsubsection{Звук и звуковые эффекты}
        \subsubsection{Музыка}
    \subsection{Описание уровней}
        \subsubsection{Общее описание дизайна уровней}
        \subsubsection{Диаграмма взаимного расположения уровней}
        \subsubsection{График введения новых объектов}

\section{Контакты}

\newpage

\end{document}