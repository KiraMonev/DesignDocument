\documentclass{article}
\usepackage{graphicx} % Required for inserting images
\usepackage[T2A]{fontenc}
\usepackage{geometry} % Managing page fields
\geometry{top=2cm, bottom=2cm, left=3cm, right=2cm}

\title{Дизайн документ}
\author{Авербух Семён, Бычкова Антонина, \\ Карташов Никита, Монёв Кирилл}


\date{Ноябрь 2024}

\begin{document}

\maketitle

\tableofcontents

\section{Введение}

\section{Концепция}
    \subsection{Введение}
    \subsection{Жанр и аудитория}
    \subsection{Основные особенности игры}
    \subsection{Описание игры}
    Главный герой — обычный парень, вынужденный противостоять алкашам, гопникам, тараканам и прочим персонажам, встречающимся на пути. Игрок перемещается по комнатам, участвуя в сражениях и исследуя окружающий мир. Игра представляет собой череду комнат и локаций, каждая из которых предлагает новые испытания.

    В начале игрок сталкивается с унылой обстановкой, но по мере прохождения и развития сюжета мир вокруг него меняется. Герой становится сильнее, открывает новые возможности и навыки. Атмосфера игры постепенно трансформируется, отражая внутреннюю эволюцию героя: от мрачных и приглушённых тонов — к насыщенным и жизнеутверждающим, что символизирует его рост и способность справляться с трудностями.
    \subsection{Предпосылки создания}
    \subsection{Платформа}

\section{Функциональная спецификация}
    \subsection{Принципы игры}
        \subsubsection{Суть игрового процесса}
        \subsubsection{Ход игры и сюжет}
    \subsection{Физическая модель}
    \subsection{Персонаж игрока}
    \subsection{Элементы игры}
    \subsection{«Искусственный интеллект»}
    \subsection{Многопользовательский режим}
    \subsection{Интерфейс пользователя}
        \subsubsection{Блок-схема}
        \subsubsection{Функциональное описание и управление}
        \subsubsection{Объекты интерфейса пользователя}
    \subsection{Графика и видео}
        \subsubsection{Общее описание}
        \subsubsection{Двумерная графика и анимация}
        \subsubsection{Трехмерная графика и анимация}
        \subsubsection{Анимационные вставки}
    \subsection{Звуки и музыка}
        \subsubsection{Общее описание}
        \subsubsection{Звук и звуковые эффекты}
        \subsubsection{Музыка}
    \subsection{Описание уровней}
        \subsubsection{Общее описание дизайна уровней}
        \subsubsection{Диаграмма взаимного расположения уровней}
        \subsubsection{График введения новых объектов}

\section{Контакты}

\newpage

\end{document}