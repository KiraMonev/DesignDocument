\documentclass{article}
\usepackage{graphicx} % Required for inserting images
\usepackage[T2A]{fontenc}
\usepackage{geometry} % Managing page fields
\geometry{top=2cm, bottom=2cm, left=3cm, right=2cm}

\title{Дизайн документ}
\author{Авербух Семён, Бычкова Антонина, \\ Карташов Никита, Монёв Кирилл}


\date{Ноябрь 2024}

\begin{document}

\maketitle

\tableofcontents

\section{Введение}
Данный дизайн-документ знакомит с концепцией и спецификацией разрабатывающейся игры. Целью документа является создание структурированного и подробного описания множества аспектов игры, начиная от самой концепции и заканчивая техническими деталями реализации.

Игра разрабатывается как 2D-проект с видом сверху, где сюжет и сражения являются основным акцентом. Вдохновением для игры послужили такие проекты, как \textit{Hotline Miami} и \textit{Dead Cells}.

Цель игры — познакомить игрока с историей о борьбе и внутреннем росте главного героя, дать возможность поучаствовать в увлекательных сражениях и самому выбирать, как будут развиваться события. Игрок получит уникальный опыт взаимодействия с динамичным игровым миром, а также сможет проверить свои интеллектуальные способности.

\section{Концепция}
    \subsection{Введение}
Игра переносит игрока в суровую, но одновременно уютную реальность российской глубинки, начинающийся с мрачной атмосферы квартиры главного героя. Одновременно с тем, как игрок проходит испытания и раскрывает сюжет, обстановка вокруг становится светлее, передавая внутреннюю трансформацию персонажа.

Основная идея игры — показать, что даже в самых трудных условиях можно найти путь к улучшению и изменению себя и своего мира.
    \subsection{Жанр и аудитория}
    \textbf{Жанр:} 2D экшн с сюжетом и элементами головоломки.
    \subsection{Основные особенности игры}
    \begin{enumerate}
    \item \textbf{Вид сверху и 2D-графика} — обеспечивает простой и понятный геймплей.
    \item \textbf{Развитие сюжета} — история героя является центральной частью игры.
    \item \textbf{Смена атмосферы} — визуальный стиль отражает прогресс игрока, от серых и мрачных тонов к более ярким и насыщенным.
    \item \textbf{Разнообразие сражений} — использование холодного и огнестрельного оружия.
    \item \textbf{Интерактивность} — возможность взаимодействовать с предметами, решать головоломки.
\end{enumerate}
    \subsection{Описание игры}
    Главный герой — обычный парень, вынужденный противостоять алкашам, гопникам, тараканам и прочим персонажам, встречающимся на пути. Игрок перемещается по комнатам, участвуя в сражениях и исследуя окружающий мир. Игра представляет собой череду комнат и локаций, каждая из которых предлагает новые испытания.

    В начале игрок сталкивается с унылой обстановкой, но по мере прохождения и развития сюжета мир вокруг него меняется. Герой становится сильнее, открывает новые возможности и навыки. Атмосфера игры постепенно трансформируется, отражая внутреннюю эволюцию героя: от мрачных и приглушённых тонов — к насыщенным и жизнеутверждающим, что символизирует его рост и способность справляться с трудностями.
    \subsection{Предпосылки создания}
    \setlength{\parindent}{1.5em}В последние годы наблюдается заметный рост интереса к 2D-играм, особенно среди инди-разработчиков. Игры с пиксельной графикой и уникальным художественным стилем привлекают внимание как новых, так и опытных игроков. Успех таких проектов, как Celeste, Hollow Knight и Katana ZERO, подтверждает, что аудитория готова поддерживать качественные 2D-игры, предлагающие интересный геймплей и захватывающую атмосферу. Помимо этого, современные игроки все чаще ищут уникальные комбинации жанров. Hotline Miami и Dead Cells представляют собой разные подходы к экшену, и их сочетание может создать новый, увлекательный опыт. Игры, которые успешно смешивают элементы различных жанров, часто становятся популярными, так как они предлагают игрокам разнообразие и новизну.
    
        \setlength{\parindent}{1.5em}Важно отметить, что создание игры, вдохновленной Hotline Miami и Dead Cells, не подразумевает прямого копирования их контента. Лицензирование и авторские права защищают конкретные механики, персонажей и сюжетные линии, но не идеи и концепции.
    \subsection{Платформа}

\section{Функциональная спецификация}
    \subsection{Принципы игры}
        \subsubsection{Суть игрового процесса}
        \subsubsection{Ход игры и сюжет}
    \subsection{Физическая модель}
    \subsection{Персонаж игрока}
    \subsection{Элементы игры}
    \subsection{«Искусственный интеллект»}
    \subsection{Многопользовательский режим}
    \subsection{Интерфейс пользователя}
        \subsubsection{Блок-схема}
        \subsubsection{Функциональное описание и управление}
        \subsubsection{Объекты интерфейса пользователя}
    \subsection{Графика и видео}
        \subsubsection{Общее описание}
        \subsubsection{Двумерная графика и анимация}
        \subsubsection{Трехмерная графика и анимация}
        \subsubsection{Анимационные вставки}
    \subsection{Звуки и музыка}
        \subsubsection{Общее описание}
        \subsubsection{Звук и звуковые эффекты}
        \subsubsection{Музыка}
    \subsection{Описание уровней}
        \subsubsection{Общее описание дизайна уровней}
        \subsubsection{Диаграмма взаимного расположения уровней}
        \subsubsection{График введения новых объектов}

\section{Контакты}

\newpage

\end{document}
