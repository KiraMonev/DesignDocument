\documentclass[12pt]{article}
\usepackage{graphicx} % Required for inserting images
\usepackage[T2A]{fontenc}
\usepackage{geometry} % Managing page fields
\geometry{top=2cm, bottom=2cm, left=3cm, right=2cm}
\usepackage{booktabs}

\title{Дизайн документ}
\author{Авербух Семён, Бычкова Антонина, \\ Карташов Никита, Монёв Кирилл}
\date{Ноябрь 2024}

\sloppy
\begin{document}

\maketitle

% Уменьшаем шрифт только для оглавления
\begingroup
\footnotesize
\tableofcontents
\endgroup

\section{Введение}
Данный дизайн-документ знакомит с концепцией и спецификацией разрабатывающейся игры. Целью документа является создание структурированного и подробного описания множества аспектов игры, начиная от самой концепции и заканчивая техническими деталями реализации.

Игра разрабатывается как 2D-проект с видом сверху, где сюжет и сражения являются основным акцентом. Вдохновением для игры послужили такие проекты, как \textit{Hotline Miami} и \textit{Dead Cells}.

Цель игры — познакомить игрока с историей о борьбе и внутреннем росте главного героя, дать возможность поучаствовать в увлекательных сражениях и самому выбирать, как будут развиваться события. Игрок получит уникальный опыт взаимодействия с динамичным игровым миром, а также сможет проверить свои интеллектуальные способности.

\section{Концепция}
    \subsection{Введение}
Игра переносит игрока в суровую, но одновременно уютную реальность российской глубинки, начинающийся с мрачной атмосферы квартиры главного героя. Одновременно с тем, как игрок проходит испытания и раскрывает сюжет, обстановка вокруг становится светлее, передавая внутреннюю трансформацию персонажа.

Основная идея игры — показать, что даже в самых трудных условиях можно найти путь к улучшению и изменению себя и своего мира.
    \subsection{Жанр и аудитория}
    \textbf{Жанр:} 2D экшн с сюжетом и элементами головоломки.
    \subsection{Основные особенности игры}
    \begin{enumerate}
    \item \textbf{Вид сверху и 2D-графика} — обеспечивает простой и понятный геймплей.
    \item \textbf{Развитие сюжета} — история героя является центральной частью игры.
    \item \textbf{Смена атмосферы} — визуальный стиль отражает прогресс игрока, от серых и мрачных тонов к более ярким и насыщенным.
    \item \textbf{Разнообразие сражений} — использование холодного и огнестрельного оружия.
    \item \textbf{Интерактивность} — возможность взаимодействовать с предметами, решать головоломки.
\end{enumerate}
    \subsection{Описание игры}
    Главный герой — обычный парень, вынужденный противостоять алкашам, гопникам, тараканам и прочим персонажам, встречающимся на пути. Игрок перемещается по комнатам, участвуя в сражениях и исследуя окружающий мир. Игра представляет собой череду комнат и локаций, каждая из которых предлагает новые испытания.

    В начале игрок сталкивается с унылой обстановкой, но по мере прохождения и развития сюжета мир вокруг него меняется. Герой становится сильнее, открывает новые возможности и навыки. Атмосфера игры постепенно трансформируется, отражая внутреннюю эволюцию героя: от мрачных и приглушённых тонов — к насыщенным и жизнеутверждающим, что символизирует его рост и способность справляться с трудностями.
    \subsection{Предпосылки создания}
    \setlength{\parindent}{1.5em}В последние годы наблюдается заметный рост интереса к 2D-играм, особенно среди инди-разработчиков. Игры с пиксельной графикой и уникальным художественным стилем привлекают внимание как новых, так и опытных игроков. Успех таких проектов, как Celeste, Hollow Knight и Katana ZERO, подтверждает, что аудитория готова поддерживать качественные 2D-игры, предлагающие интересный геймплей и захватывающую атмосферу. Помимо этого, современные игроки все чаще ищут уникальные комбинации жанров. Hotline Miami и Dead Cells представляют собой разные подходы к экшену, и их сочетание может создать новый, увлекательный опыт. Игры, которые успешно смешивают элементы различных жанров, часто становятся популярными, так как они предлагают игрокам разнообразие и новизну.
    
        \setlength{\parindent}{1.5em}Важно отметить, что создание игры, вдохновленной Hotline Miami и Dead Cells, не подразумевает прямого копирования их контента. Лицензирование и авторские права защищают конкретные механики, персонажей и сюжетные линии, но не идеи и концепции.
    \subsection{Платформа}
    Создание игры планируется на PC(Windows). Ниже перечислены минимальные и рекомендованные системные требования.
    \begin{table}[h]
        \centering
        \begin{tabular}{@{}lll@{}}
            \toprule
            \textbf {Требования} & \textbf {Минимальные} & \textbf {Рекомендуемые} \\ \midrule
            Операционная система & Windows 7 & Windows 10 \\ 
            Процессор & 2.0 GHz Dual-Core & 3.0 GHz Quad-Core \\ 
            Оперативная память (ОЗУ) & 4 GB RAM & 8 GB RAM \\ 
            Видеокарта & NVIDIA GeForce 660& NVIDIA GeForce GTX 970 \\ 
            Место на диске & 2 GB & 4 GB \\ 
            Свободное место на HDD & 2 GB & 4 GB \\ 
            CD-ROM привод & Не требуется & Не требуется \\ 
            Звуковая карта & Совместимая с DirectX & Совместимая с DirectX \\ 
            Управление & Клавиатура и мышь & Клавиатура и мышь / Gamepad \\ 
            \bottomrule
        \end{tabular}
    \end{table}

\section{Функциональная спецификация}
    \subsection{Принципы игры}
        \subsubsection{Суть игрового процесса}
        \subsubsection{Ход игры и сюжет}
        
    \subsection{Физическая модель}

        Физическая модель игры создаёт интуитивно понятный и динамичный игровой процесс. Игра сосредоточена на простоте и удобстве взаимодействия с окружением, чтобы игрок мог наслаждаться сражениями и исследованием, не отвлекаясь на сложные физические механики.
        
        \subsubsection{Перемещения}
        
        \begin{itemize}
            \item \textbf{Движение персонажа:} Главный герой может свободно перемещаться в любом направлении по локации с фиксированной скоростью. Его движения плавные, с мгновенной реакцией на команды игрока.
            \begin{itemize}
                \item \textbf{W} — движение вверх.
                \item \textbf{A} — движение влево.
                \item \textbf{S} — движение вниз.
                \item \textbf{D} — движение вправо.
            \end{itemize}
            \item \textbf{Препятствия:} Столкновения с предметами, такими как стены или мебель, приводят к остановке движения. Маленькие предметы, например, бутылки, могут быть сдвинуты или перевёрнуты.
        \end{itemize}
        
        \subsubsection{Боевые действия}
        
        \begin{itemize}
            \item \textbf{Холодное оружие:} Удары совершаются мгновенно, анимации короткие. Разные виды оружия отличаются скоростью атаки и нанесённым уроном. Например, нож быстрый, но менее мощный, а лом медленный, но разрушительный.
            \item \textbf{Огнестрельное оружие:} Стрельба осуществляется мгновенно, пули попадают точно в цель. Боеприпасы ограничены.  Это создает необходимость для игрока использовать их с умом, подстраивая свои действия под доступные патроны. Игрок может искать дополнительные боеприпасы в различных локациях, что добавляет элемент исследования и стимулирует стратегический подход к каждому сражению.
            \begin{itemize}
                \item \textbf{Пистолеты:} 
                \begin{itemize}
                    \item Идеальны для быстрых выстрелов и мобильности.
                    \item Меньшая мощность по сравнению с другими видами оружия.
                    \item Высокая скорость перезарядки.
                \end{itemize}
                \item \textbf{Дробовики:} 
                \begin{itemize}
                    \item Мощные на близких расстояниях.
                    \item Ограниченная дальнобойность.
                    \item Низкая скорострельность.
                \end{itemize}
                \item \textbf{Автоматы:} 
                \begin{itemize}
                    \item Высокая скорострельность.
                    \item Быстрое расходование патронов.
                    \item Хороши для массовых сражений на средних и близких расстояниях.
                \end{itemize}
            \end{itemize}
            \item \textbf{Враги:} 
                \begin{itemize}
                    \item При попадании враги могут быть сбиты с ног или отлететь, в зависимости от силы удара и используемого оружия.
                    \item Враги могут иметь разное количество здоровья, что влияет на их поведение в бою. Например, одни враги могут погибать от одного удара, а другие — требовать несколько попаданий.
                    \item Враги могут атаковать как в ближнем бою (с использованием оружия или без него), так и на расстоянии, используя огнестрельное оружие.
                    \item Некоторые враги могут использовать укрытия, чтобы избегать попаданий, или пытаться окружить игрока, что заставляет его быть более внимательным и постоянно двигаться.
                    \item Некоторые типы врагов обладают особыми атаками, например, могут бросать взрывчатку или устанавливать ловушки, что добавляет элемент неожиданности и стратегической сложности.
                \end{itemize}
        \end{itemize}
        
        \subsubsection{Взаимодействие с окружением}

            \begin{itemize}
                \item \textbf{Использование укрытий:} 
                \begin{itemize}
                    \item Игрок может прятаться за различными объектами на уровне, такими как мебель, стены, ящики и другие препятствия. Укрытия позволяют избегать огня врагов, защищать себя от выстрелов и подстраивать тактику нападения.
                    \item Некоторые укрытия могут быть разрушимыми, что добавляет динамичности в бою. Игрок должен учитывать, насколько долго он может оставаться за укрытием, прежде чем оно разрушится или будет уничтожено врагами.
                \end{itemize}
                
                \item \textbf{Ловушки:} 
                    Некоторые уровни могут включать в себя ловушки, такие как острые шипы или электрические проводки, которые будут наносить урон игроку.
                \item \textbf{Взаимодействие с окружающими элементами:}
                        На уровнях будут встречаться ящики, в которых можно приобрести оружие за собранные монеты. Монеты можно найти, исследуя уровни, что мотивирует игрока к более детальному изучению карты.
            \end{itemize}

        
        \subsubsection{Общая атмосфера физики}
        
        Физика мира не усложняет игровой процесс, но дополняет его. Игрок всегда чувствует контроль над героем и его взаимодействиями, что создаёт погружение в атмосферу игры. Разрушения, звуки и эффекты делают каждое столкновение или действие запоминающимся.

    \subsection{Персонаж игрока}
    \subsection{Элементы игры}
    \subsection{«Искусственный интеллект»}
            Искусственный интелект будет присущ различным противникам, в зависимости от их способностей и количества текущего здоровья, они будут по разному вести бой, убегать, взаимодействовать с предметами. Когда игрок вне поля зрения противника, то враги патрулируют местность или действуют согласно их роли и образа. 
    \subsection{Многопользовательский режим}
    \subsection{Интерфейс пользователя}
        \subsubsection{Блок-схема}
        \subsubsection{Функциональное описание и управление}
        \subsubsection{Объекты интерфейса пользователя}
    \subsection{Графика и видео}
        \subsubsection{Общее описание}
        \subsubsection{Двумерная графика и анимация}
        \subsubsection{Трехмерная графика и анимация}
        \subsubsection{Анимационные вставки}
    \subsection{Звуки и музыка}
        \subsubsection{Общее описание}
        \subsubsection{Звук и звуковые эффекты}
        \subsubsection{Музыка}
    \subsection{Описание уровней}
        \subsubsection{Общее описание дизайна уровней}
        \subsubsection{Диаграмма взаимного расположения уровней}
        \subsubsection{График введения новых объектов}

\section{Контакты}

\newpage

\end{document}
